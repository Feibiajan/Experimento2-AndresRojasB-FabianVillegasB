
\documentclass[journal]{IEEEtran}
%
\usepackage{instructivo}  % Commands for multiple choice
\graphicspath{{./}{./fig/}}

\usepackage{float}

\usepackage[natbib=true,style=trad-unsrt,backend=biber]{biblatex}

\addbibresource{literatura.bib}

% correct bad hyphenation here
\hyphenation{op-tical net-works semi-conduc-tor}


\begin{document}
\title{Experimento 04: Transistores de unión bipolar}


\author{Andrés~Arturo~Rojas-Barboza,~\IEEEmembership{Estudiante,~ITCR}
        y~Fabián~Mauricio~Villegas-Bonilla,~\IEEEmembership{Estudiante,~ITCR,}
}


% The paper headers
\markboth{EL3215 Laboratorio de Electrónica Analógica, IIS2022}%
{EL3215 Laboratorio de Electrónica Analógica}


% make the title area
\maketitle


\begin{abstract}
El informe presenta el desarollo y resultado del experimento 4 de transistores en union bipolar, donde se formaron varios circuitos con transistores bjt en su configuración npn para medir su curva caracteristicas y el resultado de sus señales de salida con diferentes corrientes en su terminal "b" para su polarización y voltajes en su terminal "c".
\end{abstract}

% Se sugiere no más de cuatro palabras o frases cortas en orden alfabético, separadas por comas, que representen su reporte
\begin{IEEEkeywords}
Diodo, tensión, corriente, junta, ruptura.
\end{IEEEkeywords}


\section{Introducción}

\IEEEPARstart{L}{os} transistores de union bipolar bjt son semiconducores de tres capas que pueden ser con materiales de tipo npn o pnp esto determinara la dirección de su corriente, para este experimento se usara solo el 2N3904 (npn). Los transistores bjt tienen tres terminales llamados colector(C), emisor (E) y base (B), donde la base sirve como un switch de polarización para permitir que la corriente en la terminal del colector. [132]
Que tienen la caracteristica de aumentar la corriente o el voltaje de una salida de tensión. Los transis


\section{Ecuaciones}
Para realizar la mediciones de voltaje en los circuitos se uso la ley de Ohm. La ecuación de la ley de Ohm es
\par
\begin{equation}
V=I\cdot R
\end{equation}


De modo que al realizar el despeje de la corriente de como resultado la ecuación

\begin{equation}
I=\frac{V}{R}
\end{equation}

\subsection{Circuitos de Medición}

El circuito de la Fig. \ref{fig:T1} sirve como referencia para obtener las curva caracteristcas por medio de diferenctes mediciones con corriente de base y voltaje de colector-emisor variables y la Fig.\ref{fig:T2} se utiliza para obtener una curva característica de los transitores en un osciloscopio.

 
\begin{figure}[H]
	\centering
	\includegraphics[width=2.5in]{circuito_T1}
	\caption{Circuito de emisor común para medir curvas caracteríscticas del BJT}
	\label{fig:T1}	
\end{figure}
\hfill

\begin{figure}[H]
	\centering
	\includegraphics[width=2.5in]{circuito_T2}
	\caption{Circuito para observar curvas características del BJT con el osciloscopio}
	\label{fig:T2}	
\end{figure}
\hfill

\subsection{Resultados Simulados}

\subsection{Resultados Experimentales}

La \ref{tabla1} muestra los valores de la resistencia reales utilizados para hacer las mediciones de la 


\vspace{5mm}

\subsection{Análisis de Resultados}

La tensión de ruptura del diodo se puede aproximar en $0.7\,$V \cite{Malik1996,Boylestad,Horowitz1989,Gray1995}. Por otro lado, la tensión de ruptura se puede considerar de $-40\,$V \cite{Floyd2008,Behzad2013,Schilling1994}. Un diodo puede considerarse como una junta de dos materiales, uno con un dopado de portadores mayoritarios mayor al otro material de la junta \cite{Pierret1994}.\\




\section{Graficando la Curva Característica del Transistor}








\section{Conclusiones}
 Por medio de la información de la fig se muestra que si se cumple la teoría donde el diodo tiene su punto de trabajo alrededor de los 0.6V a los 0.8V siendo los 0.7V el punto donde se encuentra el mayor crecimiento de corriente.
 
\section{Referencias}
R. L. BOYLESTAD, y L. NASHELSKY, 
\textit{Electrónica: Teoría de Circuitos
y Dispositivos Electrónicos}. PEARSON EDUCACIÓN, México, 2009
%%%%%%%%%%%%%%%%%%%%%%%%%%%%%%%%%%%%%%%%%%%%%%%%%%%%%%%%%%%%%%%%%%%%%%%%%%%%%%%%%%%%
%%%%%%%%%%%%%%%%%%%%%%%%%%%%%%%%%%%%%%%%%%%%%%%%%%%%%%%%%%%%%%%%%%%%%%%%%%%%%%%%%%%%
\appendices
\section{Demostración de la Ley de Ohm}
El texto relacionado al apéndice va aquí.

\section{Cálculos de polarización CD}
El texto relacionado al apéndice va aquí.

\end{document}






