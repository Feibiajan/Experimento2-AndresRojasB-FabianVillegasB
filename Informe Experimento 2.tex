
\documentclass[journal]{IEEEtran}
%
\usepackage{instructivo}  % Commands for multiple choice
\graphicspath{{./}{./fig/}}

\usepackage{float}

\usepackage[natbib=true,style=trad-unsrt,backend=biber]{biblatex}

\addbibresource{literatura.bib}

% correct bad hyphenation here
\hyphenation{op-tical net-works semi-conduc-tor}


\begin{document}
\title{Experimento 04: Transistores de unión bipolar}


\author{Andrés~Arturo~Rojas-Barboza,~\IEEEmembership{Estudiante,~ITCR}
        y~Fabián~Mauricio~Villegas-Bonilla,~\IEEEmembership{Estudiante,~ITCR,}
}


% The paper headers
\markboth{EL3215 Laboratorio de Electrónica Analógica, IIS2022}%
{EL3215 Laboratorio de Electrónica Analógica}


% make the title area
\maketitle


\begin{abstract}
El informe presenta el desarollo y resultado del experimento 4 de transistores en union bipolar, donde se formaron varios circuitos con transistores bjt en su configuración npn para medir su curva caracteristicas y el resultado de sus señales de salida con diferentes corrientes en su terminal "b" para su polarización y voltajes en su terminal "c".
\end{abstract}

% Se sugiere no más de cuatro palabras o frases cortas en orden alfabético, separadas por comas, que representen su reporte
\begin{IEEEkeywords}
Diodo, tensión, corriente, junta, ruptura.
\end{IEEEkeywords}


\section{Introducción}

\IEEEPARstart{L}{os} transistores de union bipolar bjt son semiconducores de tres capas que pueden ser con materiales de tipo npn o pnp esto determinara la dirección de su corriente, para este experimento se usara solo el 2N3904 (npn). Los transistores bjt tienen tres terminales llamados colector(C), emisor (E) y base (B), donde la base sirve como un switch de polarización para permitir que la corriente en la terminal del colector. [132]

Los transistores bjt tiene la función de crear una ganancia de corriente para el voltaje de salida del circuito. Para que el bjt este encendido (polarizado) este necesita de una fuente de corriente directa (CD) en su terminales de base y colector en caso de los npn, de modo que al agregarle una fuente de corriente alterna su señal dependera de como este modificado el circuito con su transistor

\section{Experimenacion}
\subsection{Ecuaciones}
Para realizar la mediciones de voltaje en los circuitos se uso la ley de Ohm. La ecuación de la ley de Ohm es
\par
\begin{equation}
V=I\cdot R
\end{equation}


De modo que al realizar el despeje de la corriente de como resultado la ecuación

\begin{equation}
I=\frac{V}{R}
\end{equation}

\subsection{Circuitos de Medición}

El circuito de la Fig. \ref{fig:T1} sirve como referencia para obtener las curva caracteristcas por medio de diferenctes mediciones con corriente de base y voltaje de colector-emisor variables y la Fig.\ref{fig:T2} se utiliza para obtener una curva característica de los transitores en un osciloscopio.

 
\begin{figure}[H]
	\centering
	\includegraphics[width=2.5in]{circuito_T1}
	\caption{Circuito de emisor común para medir curvas caracteríscticas del BJT}
	\label{fig:T1}	
\end{figure}
\hfill

\begin{figure}[H]
	\centering
	\includegraphics[width=2.5in]{circuito_T2}
	\caption{Circuito para observar curvas características del BJT con el osciloscopio}
	\label{fig:T2}	
\end{figure}
\hfill

\subsection{Resultados Simulados}

\begin{figure}[H]
	\centering
	\includegraphics[width=2.5in]{tabla1}
	\caption{Tabla 2: Resultados simulados}
	\label{fig:Ta1}	
\end{figure}
\hfill

\begin{figure}[H]
	\centering
	\includegraphics[width=2.5in]{tabla2}
	\caption{Ganancia en corriente Bcd}
	\label{fig:Ta2}	
\end{figure}
\hfill
\subsection{Resultados Experimentales}


\vspace{5mm}

\subsection{Análisis de Resultados}

 \cite{Boylestad}. 




\section{Graficando la Curva Característica del Transistor}








\section{Conclusiones}
 
 
\section{Referencias}
R. L. BOYLESTAD, y L. NASHELSKY, 
\textit{Electrónica: Teoría de Circuitos
y Dispositivos Electrónicos}. PEARSON EDUCACIÓN, México, 2009


\appendices
\section{Demostración de la Ley de Ohm}
El texto relacionado al apéndice va aquí.


\end{document}






